\chapter{Lexical analysis\label{lexical}}

A Python program is read by a \emph{parser}.  Input to the parser is a
stream of \emph{tokens}, generated by the \emph{lexical analyzer}.  This
chapter describes how the lexical analyzer breaks a file into tokens.
\index{lexical analysis}
\index{parser}
\index{token}

Python uses the 7-bit \ASCII{} character set for program text.
\versionadded[An encoding declaration can be used to indicate that 
string literals and comments use an encoding different from ASCII]{2.3}
For compatibility with older versions, Python only warns if it finds
8-bit characters; those warnings should be corrected by either declaring
an explicit encoding, or using escape sequences if those bytes are binary
data, instead of characters.


The run-time character set depends on the I/O devices connected to the
program but is generally a superset of \ASCII.

\strong{Future compatibility note:} It may be tempting to assume that the
character set for 8-bit characters is ISO Latin-1 (an \ASCII{}
superset that covers most western languages that use the Latin
alphabet), but it is possible that in the future Unicode text editors
will become common.  These generally use the UTF-8 encoding, which is
also an \ASCII{} superset, but with very different use for the
characters with ordinals 128-255.  While there is no consensus on this
subject yet, it is unwise to assume either Latin-1 or UTF-8, even
though the current implementation appears to favor Latin-1.  This
applies both to the source character set and the run-time character
set.


\section{Line structure\label{line-structure}}

A Python program is divided into a number of \emph{logical lines}.
\index{line structure}


\subsection{Logical lines\label{logical}}

The end of
a logical line is represented by the token NEWLINE.  Statements cannot
cross logical line boundaries except where NEWLINE is allowed by the
syntax (e.g., between statements in compound statements).
A logical line is constructed from one or more \emph{physical lines}
by following the explicit or implicit \emph{line joining} rules.
\index{logical line}
\index{physical line}
\index{line joining}
\index{NEWLINE token}


\subsection{Physical lines\label{physical}}

A physical line is a sequence of characters terminated by an end-of-line
sequence.  In source files, any of the standard platform line
termination sequences can be used - the \UNIX{} form using \ASCII{} LF
(linefeed), the Windows form using the \ASCII{} sequence CR LF (return
followed by linefeed), or the Macintosh form using the \ASCII{} CR
(return) character.  All of these forms can be used equally, regardless
of platform.

When embedding Python, source code strings should be passed to Python
APIs using the standard C conventions for newline characters (the
\code{\e n} character, representing \ASCII{} LF, is the line
terminator).


\subsection{Comments\label{comments}}

A comment starts with a hash character (\code{\#}) that is not part of
a string literal, and ends at the end of the physical line.  A comment
signifies the end of the logical line unless the implicit line joining
rules are invoked.
Comments are ignored by the syntax; they are not tokens.
\index{comment}
\index{hash character}


\subsection{Encoding declarations\label{encodings}}
\index{source character set}
\index{encodings}

If a comment in the first or second line of the Python script matches
the regular expression \regexp{coding[=:]\e s*([-\e w.]+)}, this comment is
processed as an encoding declaration; the first group of this
expression names the encoding of the source code file. The recommended
forms of this expression are

\begin{verbatim}
# -*- coding: <encoding-name> -*-
\end{verbatim}

which is recognized also by GNU Emacs, and

\begin{verbatim}
# vim:fileencoding=<encoding-name>
\end{verbatim}

which is recognized by Bram Moolenaar's VIM. In addition, if the first
bytes of the file are the UTF-8 byte-order mark
(\code{'\e xef\e xbb\e xbf'}), the declared file encoding is UTF-8
(this is supported, among others, by Microsoft's \program{notepad}).

If an encoding is declared, the encoding name must be recognized by
Python. % XXX there should be a list of supported encodings.
The encoding is used for all lexical analysis, in particular to find
the end of a string, and to interpret the contents of Unicode literals.
String literals are converted to Unicode for syntactical analysis,
then converted back to their original encoding before interpretation
starts. The encoding declaration must appear on a line of its own.

\subsection{Explicit line joining\label{explicit-joining}}

Two or more physical lines may be joined into logical lines using
backslash characters (\code{\e}), as follows: when a physical line ends
in a backslash that is not part of a string literal or comment, it is
joined with the following forming a single logical line, deleting the
backslash and the following end-of-line character.  For example:
\index{physical line}
\index{line joining}
\index{line continuation}
\index{backslash character}
%
\begin{verbatim}
if 1900 < year < 2100 and 1 <= month <= 12 \
   and 1 <= day <= 31 and 0 <= hour < 24 \
   and 0 <= minute < 60 and 0 <= second < 60:   # Looks like a valid date
        return 1
\end{verbatim}

A line ending in a backslash cannot carry a comment.  A backslash does
not continue a comment.  A backslash does not continue a token except
for string literals (i.e., tokens other than string literals cannot be
split across physical lines using a backslash).  A backslash is
illegal elsewhere on a line outside a string literal.


\subsection{Implicit line joining\label{implicit-joining}}

Expressions in parentheses, square brackets or curly braces can be
split over more than one physical line without using backslashes.
For example:

\begin{verbatim}
month_names = ['Januari', 'Februari', 'Maart',      # These are the
               'April',   'Mei',      'Juni',       # Dutch names
               'Juli',    'Augustus', 'September',  # for the months
               'Oktober', 'November', 'December']   # of the year
\end{verbatim}

Implicitly continued lines can carry comments.  The indentation of the
continuation lines is not important.  Blank continuation lines are
allowed.  There is no NEWLINE token between implicit continuation
lines.  Implicitly continued lines can also occur within triple-quoted
strings (see below); in that case they cannot carry comments.


\subsection{Blank lines \label{blank-lines}}

\index{blank line}
A logical line that contains only spaces, tabs, formfeeds and possibly
a comment, is ignored (i.e., no NEWLINE token is generated).  During
interactive input of statements, handling of a blank line may differ
depending on the implementation of the read-eval-print loop.  In the
standard implementation, an entirely blank logical line (i.e.\ one
containing not even whitespace or a comment) terminates a multi-line
statement.


\subsection{Indentation\label{indentation}}

Leading whitespace (spaces and tabs) at the beginning of a logical
line is used to compute the indentation level of the line, which in
turn is used to determine the grouping of statements.
\index{indentation}
\index{whitespace}
\index{leading whitespace}
\index{space}
\index{tab}
\index{grouping}
\index{statement grouping}

First, tabs are replaced (from left to right) by one to eight spaces
such that the total number of characters up to and including the
replacement is a multiple of
eight (this is intended to be the same rule as used by \UNIX).  The
total number of spaces preceding the first non-blank character then
determines the line's indentation.  Indentation cannot be split over
multiple physical lines using backslashes; the whitespace up to the
first backslash determines the indentation.

\strong{Cross-platform compatibility note:} because of the nature of
text editors on non-UNIX platforms, it is unwise to use a mixture of
spaces and tabs for the indentation in a single source file.  It
should also be noted that different platforms may explicitly limit the
maximum indentation level.

A formfeed character may be present at the start of the line; it will
be ignored for the indentation calculations above.  Formfeed
characters occurring elsewhere in the leading whitespace have an
undefined effect (for instance, they may reset the space count to
zero).

The indentation levels of consecutive lines are used to generate
INDENT and DEDENT tokens, using a stack, as follows.
\index{INDENT token}
\index{DEDENT token}

Before the first line of the file is read, a single zero is pushed on
the stack; this will never be popped off again.  The numbers pushed on
the stack will always be strictly increasing from bottom to top.  At
the beginning of each logical line, the line's indentation level is
compared to the top of the stack.  If it is equal, nothing happens.
If it is larger, it is pushed on the stack, and one INDENT token is
generated.  If it is smaller, it \emph{must} be one of the numbers
occurring on the stack; all numbers on the stack that are larger are
popped off, and for each number popped off a DEDENT token is
generated.  At the end of the file, a DEDENT token is generated for
each number remaining on the stack that is larger than zero.

Here is an example of a correctly (though confusingly) indented piece
of Python code:

\begin{verbatim}
def perm(l):
        # Compute the list of all permutations of l
    if len(l) <= 1:
                  return [l]
    r = []
    for i in range(len(l)):
             s = l[:i] + l[i+1:]
             p = perm(s)
             for x in p:
              r.append(l[i:i+1] + x)
    return r
\end{verbatim}

The following example shows various indentation errors:

\begin{verbatim}
 def perm(l):                       # error: first line indented
for i in range(len(l)):             # error: not indented
    s = l[:i] + l[i+1:]
        p = perm(l[:i] + l[i+1:])   # error: unexpected indent
        for x in p:
                r.append(l[i:i+1] + x)
            return r                # error: inconsistent dedent
\end{verbatim}

(Actually, the first three errors are detected by the parser; only the
last error is found by the lexical analyzer --- the indentation of
\code{return r} does not match a level popped off the stack.)


\subsection{Whitespace between tokens\label{whitespace}}

Except at the beginning of a logical line or in string literals, the
whitespace characters space, tab and formfeed can be used
interchangeably to separate tokens.  Whitespace is needed between two
tokens only if their concatenation could otherwise be interpreted as a
different token (e.g., ab is one token, but a b is two tokens).


\section{Other tokens\label{other-tokens}}

Besides NEWLINE, INDENT and DEDENT, the following categories of tokens
exist: \emph{identifiers}, \emph{keywords}, \emph{literals},
\emph{operators}, and \emph{delimiters}.
Whitespace characters (other than line terminators, discussed earlier)
are not tokens, but serve to delimit tokens.
Where
ambiguity exists, a token comprises the longest possible string that
forms a legal token, when read from left to right.


\section{Identifiers and keywords\label{identifiers}}

Identifiers (also referred to as \emph{names}) are described by the following
lexical definitions:
\index{identifier}
\index{name}

\begin{productionlist}
  \production{identifier}
             {(\token{letter}|"_") (\token{letter} | \token{digit} | "_")*}
  \production{letter}
             {\token{lowercase} | \token{uppercase}}
  \production{lowercase}
             {"a"..."z"}
  \production{uppercase}
             {"A"..."Z"}
  \production{digit}
             {"0"..."9"}
\end{productionlist}

Identifiers are unlimited in length.  Case is significant.


\subsection{Keywords\label{keywords}}

The following identifiers are used as reserved words, or
\emph{keywords} of the language, and cannot be used as ordinary
identifiers.  They must be spelled exactly as written here:%
\index{keyword}%
\index{reserved word}

\begin{verbatim}
and       del       from      not       while    
as        elif      global    or        with     
assert    else      if        pass      yield    
break     except    import    print              
class     exec      in        raise              
continue  finally   is        return             
def       for       lambda    try 
\end{verbatim}

% When adding keywords, use reswords.py for reformatting

\versionchanged[\constant{None} became a constant and is now
recognized by the compiler as a name for the built-in object
\constant{None}.  Although it is not a keyword, you cannot assign
a different object to it]{2.4}

\versionchanged[Both \keyword{as} and \keyword{with} are only recognized
when the \code{with_statement} future feature has been enabled.
It will always be enabled in Python 2.6.  See section~\ref{with} for
details.  Note that using \keyword{as} and \keyword{with} as identifiers
will always issue a warning, even when the \code{with_statement} future
directive is not in effect]{2.5}


\subsection{Reserved classes of identifiers\label{id-classes}}

Certain classes of identifiers (besides keywords) have special
meanings.  These classes are identified by the patterns of leading and
trailing underscore characters:

\begin{description}

\item[\code{_*}]
  Not imported by \samp{from \var{module} import *}.  The special
  identifier \samp{_} is used in the interactive interpreter to store
  the result of the last evaluation; it is stored in the
  \module{__builtin__} module.  When not in interactive mode, \samp{_}
  has no special meaning and is not defined.
  See section~\ref{import}, ``The \keyword{import} statement.''

  \note{The name \samp{_} is often used in conjunction with
  internationalization; refer to the documentation for the
  \ulink{\module{gettext} module}{../lib/module-gettext.html} for more
  information on this convention.}

\item[\code{__*__}]
  System-defined names.  These names are defined by the interpreter
  and its implementation (including the standard library);
  applications should not expect to define additional names using this
  convention.  The set of names of this class defined by Python may be
  extended in future versions.
  See section~\ref{specialnames}, ``Special method names.''

\item[\code{__*}]
  Class-private names.  Names in this category, when used within the
  context of a class definition, are re-written to use a mangled form
  to help avoid name clashes between ``private'' attributes of base
  and derived classes.
  See section~\ref{atom-identifiers}, ``Identifiers (Names).''

\end{description}


\section{Literals\label{literals}}

Literals are notations for constant values of some built-in types.
\index{literal}
\index{constant}


\subsection{String literals\label{strings}}

String literals are described by the following lexical definitions:
\index{string literal}

\index{ASCII@\ASCII}
\begin{productionlist}
  \production{stringliteral}
             {[\token{stringprefix}](\token{shortstring} | \token{longstring})}
  \production{stringprefix}
             {"r" | "u" | "ur" | "R" | "U" | "UR" | "Ur" | "uR"}
  \production{shortstring}
             {"'" \token{shortstringitem}* "'"
              | '"' \token{shortstringitem}* '"'}
  \production{longstring}
             {"'''" \token{longstringitem}* "'''"}
  \productioncont{| '"""' \token{longstringitem}* '"""'}
  \production{shortstringitem}
             {\token{shortstringchar} | \token{escapeseq}}
  \production{longstringitem}
             {\token{longstringchar} | \token{escapeseq}}
  \production{shortstringchar}
             {<any source character except "\e" or newline or the quote>}
  \production{longstringchar}
             {<any source character except "\e">}
  \production{escapeseq}
             {"\e" <any ASCII character>}
\end{productionlist}

One syntactic restriction not indicated by these productions is that
whitespace is not allowed between the \grammartoken{stringprefix} and
the rest of the string literal. The source character set is defined
by the encoding declaration; it is \ASCII{} if no encoding declaration
is given in the source file; see section~\ref{encodings}.

\index{triple-quoted string}
\index{Unicode Consortium}
\index{string!Unicode}
In plain English: String literals can be enclosed in matching single
quotes (\code{'}) or double quotes (\code{"}).  They can also be
enclosed in matching groups of three single or double quotes (these
are generally referred to as \emph{triple-quoted strings}).  The
backslash (\code{\e}) character is used to escape characters that
otherwise have a special meaning, such as newline, backslash itself,
or the quote character.  String literals may optionally be prefixed
with a letter \character{r} or \character{R}; such strings are called
\dfn{raw strings}\index{raw string} and use different rules for interpreting
backslash escape sequences.  A prefix of \character{u} or \character{U}
makes the string a Unicode string.  Unicode strings use the Unicode character
set as defined by the Unicode Consortium and ISO~10646.  Some additional
escape sequences, described below, are available in Unicode strings.
The two prefix characters may be combined; in this case, \character{u} must
appear before \character{r}.

In triple-quoted strings,
unescaped newlines and quotes are allowed (and are retained), except
that three unescaped quotes in a row terminate the string.  (A
``quote'' is the character used to open the string, i.e. either
\code{'} or \code{"}.)

Unless an \character{r} or \character{R} prefix is present, escape
sequences in strings are interpreted according to rules similar
to those used by Standard C.  The recognized escape sequences are:
\index{physical line}
\index{escape sequence}
\index{Standard C}
\index{C}

\begin{tableiii}{l|l|c}{code}{Escape Sequence}{Meaning}{Notes}
\lineiii{\e\var{newline}} {Ignored}{}
\lineiii{\e\e}	{Backslash (\code{\e})}{}
\lineiii{\e'}	{Single quote (\code{'})}{}
\lineiii{\e"}	{Double quote (\code{"})}{}
\lineiii{\e a}	{\ASCII{} Bell (BEL)}{}
\lineiii{\e b}	{\ASCII{} Backspace (BS)}{}
\lineiii{\e f}	{\ASCII{} Formfeed (FF)}{}
\lineiii{\e n}	{\ASCII{} Linefeed (LF)}{}
\lineiii{\e N\{\var{name}\}}
        {Character named \var{name} in the Unicode database (Unicode only)}{}
\lineiii{\e r}	{\ASCII{} Carriage Return (CR)}{}
\lineiii{\e t}	{\ASCII{} Horizontal Tab (TAB)}{}
\lineiii{\e u\var{xxxx}}
        {Character with 16-bit hex value \var{xxxx} (Unicode only)}{(1)}
\lineiii{\e U\var{xxxxxxxx}}
        {Character with 32-bit hex value \var{xxxxxxxx} (Unicode only)}{(2)}
\lineiii{\e v}	{\ASCII{} Vertical Tab (VT)}{}
\lineiii{\e\var{ooo}} {Character with octal value \var{ooo}}{(3,5)}
\lineiii{\e x\var{hh}} {Character with hex value \var{hh}}{(4,5)}
\end{tableiii}
\index{ASCII@\ASCII}

\noindent
Notes:

\begin{itemize}
\item[(1)]
  Individual code units which form parts of a surrogate pair can be
  encoded using this escape sequence.
\item[(2)]
  Any Unicode character can be encoded this way, but characters
  outside the Basic Multilingual Plane (BMP) will be encoded using a
  surrogate pair if Python is compiled to use 16-bit code units (the
  default).  Individual code units which form parts of a surrogate
  pair can be encoded using this escape sequence.
\item[(3)]
  As in Standard C, up to three octal digits are accepted.
\item[(4)]
  Unlike in Standard C, exactly two hex digits are required.
\item[(5)]
  In a string literal, hexadecimal and octal escapes denote the
  byte with the given value; it is not necessary that the byte
  encodes a character in the source character set. In a Unicode
  literal, these escapes denote a Unicode character with the given
  value.
\end{itemize}


Unlike Standard \index{unrecognized escape sequence}C,
all unrecognized escape sequences are left in the string unchanged,
i.e., \emph{the backslash is left in the string}.  (This behavior is
useful when debugging: if an escape sequence is mistyped, the
resulting output is more easily recognized as broken.)  It is also
important to note that the escape sequences marked as ``(Unicode
only)'' in the table above fall into the category of unrecognized
escapes for non-Unicode string literals.

When an \character{r} or \character{R} prefix is present, a character
following a backslash is included in the string without change, and \emph{all
backslashes are left in the string}.  For example, the string literal
\code{r"\e n"} consists of two characters: a backslash and a lowercase
\character{n}.  String quotes can be escaped with a backslash, but the
backslash remains in the string; for example, \code{r"\e""} is a valid string
literal consisting of two characters: a backslash and a double quote;
\code{r"\e"} is not a valid string literal (even a raw string cannot
end in an odd number of backslashes).  Specifically, \emph{a raw
string cannot end in a single backslash} (since the backslash would
escape the following quote character).  Note also that a single
backslash followed by a newline is interpreted as those two characters
as part of the string, \emph{not} as a line continuation.

When an \character{r} or \character{R} prefix is used in conjunction
with a \character{u} or \character{U} prefix, then the \code{\e uXXXX}
and \code{\e UXXXXXXXX} escape sequences are processed while 
\emph{all other backslashes are left in the string}.
For example, the string literal
\code{ur"\e{}u0062\e n"} consists of three Unicode characters: `LATIN
SMALL LETTER B', `REVERSE SOLIDUS', and `LATIN SMALL LETTER N'.
Backslashes can be escaped with a preceding backslash; however, both
remain in the string.  As a result, \code{\e uXXXX} escape sequences
are only recognized when there are an odd number of backslashes.

\subsection{String literal concatenation\label{string-catenation}}

Multiple adjacent string literals (delimited by whitespace), possibly
using different quoting conventions, are allowed, and their meaning is
the same as their concatenation.  Thus, \code{"hello" 'world'} is
equivalent to \code{"helloworld"}.  This feature can be used to reduce
the number of backslashes needed, to split long strings conveniently
across long lines, or even to add comments to parts of strings, for
example:

\begin{verbatim}
re.compile("[A-Za-z_]"       # letter or underscore
           "[A-Za-z0-9_]*"   # letter, digit or underscore
          )
\end{verbatim}

Note that this feature is defined at the syntactical level, but
implemented at compile time.  The `+' operator must be used to
concatenate string expressions at run time.  Also note that literal
concatenation can use different quoting styles for each component
(even mixing raw strings and triple quoted strings).


\subsection{Numeric literals\label{numbers}}

There are four types of numeric literals: plain integers, long
integers, floating point numbers, and imaginary numbers.  There are no
complex literals (complex numbers can be formed by adding a real
number and an imaginary number).
\index{number}
\index{numeric literal}
\index{integer literal}
\index{plain integer literal}
\index{long integer literal}
\index{floating point literal}
\index{hexadecimal literal}
\index{octal literal}
\index{decimal literal}
\index{imaginary literal}
\index{complex!literal}

Note that numeric literals do not include a sign; a phrase like
\code{-1} is actually an expression composed of the unary operator
`\code{-}' and the literal \code{1}.


\subsection{Integer and long integer literals\label{integers}}

Integer and long integer literals are described by the following
lexical definitions:

\begin{productionlist}
  \production{longinteger}
             {\token{integer} ("l" | "L")}
  \production{integer}
             {\token{decimalinteger} | \token{octinteger} | \token{hexinteger}}
  \production{decimalinteger}
             {\token{nonzerodigit} \token{digit}* | "0"}
  \production{octinteger}
             {"0" \token{octdigit}+}
  \production{hexinteger}
             {"0" ("x" | "X") \token{hexdigit}+}
  \production{nonzerodigit}
             {"1"..."9"}
  \production{octdigit}
             {"0"..."7"}
  \production{hexdigit}
             {\token{digit} | "a"..."f" | "A"..."F"}
\end{productionlist}

Although both lower case \character{l} and upper case \character{L} are
allowed as suffix for long integers, it is strongly recommended to always
use \character{L}, since the letter \character{l} looks too much like the
digit \character{1}.

Plain integer literals that are above the largest representable plain
integer (e.g., 2147483647 when using 32-bit arithmetic) are accepted
as if they were long integers instead.\footnote{In versions of Python
prior to 2.4, octal and hexadecimal literals in the range just above
the largest representable plain integer but below the largest unsigned
32-bit number (on a machine using 32-bit arithmetic), 4294967296, were
taken as the negative plain integer obtained by subtracting 4294967296
from their unsigned value.}  There is no limit for long integer
literals apart from what can be stored in available memory.

Some examples of plain integer literals (first row) and long integer
literals (second and third rows):

\begin{verbatim}
7     2147483647                        0177
3L    79228162514264337593543950336L    0377L   0x100000000L
      79228162514264337593543950336             0xdeadbeef						    
\end{verbatim}


\subsection{Floating point literals\label{floating}}

Floating point literals are described by the following lexical
definitions:

\begin{productionlist}
  \production{floatnumber}
             {\token{pointfloat} | \token{exponentfloat}}
  \production{pointfloat}
             {[\token{intpart}] \token{fraction} | \token{intpart} "."}
  \production{exponentfloat}
             {(\token{intpart} | \token{pointfloat})
              \token{exponent}}
  \production{intpart}
             {\token{digit}+}
  \production{fraction}
             {"." \token{digit}+}
  \production{exponent}
             {("e" | "E") ["+" | "-"] \token{digit}+}
\end{productionlist}

Note that the integer and exponent parts of floating point numbers
can look like octal integers, but are interpreted using radix 10.  For
example, \samp{077e010} is legal, and denotes the same number
as \samp{77e10}.
The allowed range of floating point literals is
implementation-dependent.
Some examples of floating point literals:

\begin{verbatim}
3.14    10.    .001    1e100    3.14e-10    0e0
\end{verbatim}

Note that numeric literals do not include a sign; a phrase like
\code{-1} is actually an expression composed of the unary operator
\code{-} and the literal \code{1}.


\subsection{Imaginary literals\label{imaginary}}

Imaginary literals are described by the following lexical definitions:

\begin{productionlist}
  \production{imagnumber}{(\token{floatnumber} | \token{intpart}) ("j" | "J")}
\end{productionlist}

An imaginary literal yields a complex number with a real part of
0.0.  Complex numbers are represented as a pair of floating point
numbers and have the same restrictions on their range.  To create a
complex number with a nonzero real part, add a floating point number
to it, e.g., \code{(3+4j)}.  Some examples of imaginary literals:

\begin{verbatim}
3.14j   10.j    10j     .001j   1e100j  3.14e-10j 
\end{verbatim}


\section{Operators\label{operators}}

The following tokens are operators:
\index{operators}

\begin{verbatim}
+       -       *       **      /       //      %
<<      >>      &       |       ^       ~
<       >       <=      >=      ==      !=      <>
\end{verbatim}

The comparison operators \code{<>} and \code{!=} are alternate
spellings of the same operator.  \code{!=} is the preferred spelling;
\code{<>} is obsolescent.


\section{Delimiters\label{delimiters}}

The following tokens serve as delimiters in the grammar:
\index{delimiters}

\begin{verbatim}
(       )       [       ]       {       }      @
,       :       .       `       =       ;
+=      -=      *=      /=      //=     %=
&=      |=      ^=      >>=     <<=     **=
\end{verbatim}

The period can also occur in floating-point and imaginary literals.  A
sequence of three periods has a special meaning as an ellipsis in slices.
The second half of the list, the augmented assignment operators, serve
lexically as delimiters, but also perform an operation.

The following printing \ASCII{} characters have special meaning as part
of other tokens or are otherwise significant to the lexical analyzer:

\begin{verbatim}
'       "       #       \
\end{verbatim}

The following printing \ASCII{} characters are not used in Python.  Their
occurrence outside string literals and comments is an unconditional
error:
\index{ASCII@\ASCII}

\begin{verbatim}
$       ?
\end{verbatim}
